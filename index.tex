% Options for packages loaded elsewhere
\PassOptionsToPackage{unicode}{hyperref}
\PassOptionsToPackage{hyphens}{url}
\PassOptionsToPackage{dvipsnames,svgnames,x11names}{xcolor}
%
\documentclass[
]{agujournal2019}

\usepackage{amsmath,amssymb}
\usepackage{iftex}
\ifPDFTeX
  \usepackage[T1]{fontenc}
  \usepackage[utf8]{inputenc}
  \usepackage{textcomp} % provide euro and other symbols
\else % if luatex or xetex
  \usepackage{unicode-math}
  \defaultfontfeatures{Scale=MatchLowercase}
  \defaultfontfeatures[\rmfamily]{Ligatures=TeX,Scale=1}
\fi
\usepackage{lmodern}
\ifPDFTeX\else  
    % xetex/luatex font selection
\fi
% Use upquote if available, for straight quotes in verbatim environments
\IfFileExists{upquote.sty}{\usepackage{upquote}}{}
\IfFileExists{microtype.sty}{% use microtype if available
  \usepackage[]{microtype}
  \UseMicrotypeSet[protrusion]{basicmath} % disable protrusion for tt fonts
}{}
\makeatletter
\@ifundefined{KOMAClassName}{% if non-KOMA class
  \IfFileExists{parskip.sty}{%
    \usepackage{parskip}
  }{% else
    \setlength{\parindent}{0pt}
    \setlength{\parskip}{6pt plus 2pt minus 1pt}}
}{% if KOMA class
  \KOMAoptions{parskip=half}}
\makeatother
\usepackage{xcolor}
\setlength{\emergencystretch}{3em} % prevent overfull lines
\setcounter{secnumdepth}{5}
% Make \paragraph and \subparagraph free-standing
\makeatletter
\ifx\paragraph\undefined\else
  \let\oldparagraph\paragraph
  \renewcommand{\paragraph}{
    \@ifstar
      \xxxParagraphStar
      \xxxParagraphNoStar
  }
  \newcommand{\xxxParagraphStar}[1]{\oldparagraph*{#1}\mbox{}}
  \newcommand{\xxxParagraphNoStar}[1]{\oldparagraph{#1}\mbox{}}
\fi
\ifx\subparagraph\undefined\else
  \let\oldsubparagraph\subparagraph
  \renewcommand{\subparagraph}{
    \@ifstar
      \xxxSubParagraphStar
      \xxxSubParagraphNoStar
  }
  \newcommand{\xxxSubParagraphStar}[1]{\oldsubparagraph*{#1}\mbox{}}
  \newcommand{\xxxSubParagraphNoStar}[1]{\oldsubparagraph{#1}\mbox{}}
\fi
\makeatother


\providecommand{\tightlist}{%
  \setlength{\itemsep}{0pt}\setlength{\parskip}{0pt}}\usepackage{longtable,booktabs,array}
\usepackage{calc} % for calculating minipage widths
% Correct order of tables after \paragraph or \subparagraph
\usepackage{etoolbox}
\makeatletter
\patchcmd\longtable{\par}{\if@noskipsec\mbox{}\fi\par}{}{}
\makeatother
% Allow footnotes in longtable head/foot
\IfFileExists{footnotehyper.sty}{\usepackage{footnotehyper}}{\usepackage{footnote}}
\makesavenoteenv{longtable}
\usepackage{graphicx}
\makeatletter
\newsavebox\pandoc@box
\newcommand*\pandocbounded[1]{% scales image to fit in text height/width
  \sbox\pandoc@box{#1}%
  \Gscale@div\@tempa{\textheight}{\dimexpr\ht\pandoc@box+\dp\pandoc@box\relax}%
  \Gscale@div\@tempb{\linewidth}{\wd\pandoc@box}%
  \ifdim\@tempb\p@<\@tempa\p@\let\@tempa\@tempb\fi% select the smaller of both
  \ifdim\@tempa\p@<\p@\scalebox{\@tempa}{\usebox\pandoc@box}%
  \else\usebox{\pandoc@box}%
  \fi%
}
% Set default figure placement to htbp
\def\fps@figure{htbp}
\makeatother
% definitions for citeproc citations
\NewDocumentCommand\citeproctext{}{}
\NewDocumentCommand\citeproc{mm}{%
  \begingroup\def\citeproctext{#2}\cite{#1}\endgroup}
\makeatletter
 % allow citations to break across lines
 \let\@cite@ofmt\@firstofone
 % avoid brackets around text for \cite:
 \def\@biblabel#1{}
 \def\@cite#1#2{{#1\if@tempswa , #2\fi}}
\makeatother
\newlength{\cslhangindent}
\setlength{\cslhangindent}{1.5em}
\newlength{\csllabelwidth}
\setlength{\csllabelwidth}{3em}
\newenvironment{CSLReferences}[2] % #1 hanging-indent, #2 entry-spacing
 {\begin{list}{}{%
  \setlength{\itemindent}{0pt}
  \setlength{\leftmargin}{0pt}
  \setlength{\parsep}{0pt}
  % turn on hanging indent if param 1 is 1
  \ifodd #1
   \setlength{\leftmargin}{\cslhangindent}
   \setlength{\itemindent}{-1\cslhangindent}
  \fi
  % set entry spacing
  \setlength{\itemsep}{#2\baselineskip}}}
 {\end{list}}
\usepackage{calc}
\newcommand{\CSLBlock}[1]{\hfill\break\parbox[t]{\linewidth}{\strut\ignorespaces#1\strut}}
\newcommand{\CSLLeftMargin}[1]{\parbox[t]{\csllabelwidth}{\strut#1\strut}}
\newcommand{\CSLRightInline}[1]{\parbox[t]{\linewidth - \csllabelwidth}{\strut#1\strut}}
\newcommand{\CSLIndent}[1]{\hspace{\cslhangindent}#1}

\usepackage{url} %this package should fix any errors with URLs in refs.
\usepackage{lineno}
\usepackage[inline]{trackchanges} %for better track changes. finalnew option will compile document with changes incorporated.
\usepackage{soul}
\linenumbers
\makeatletter
\@ifpackageloaded{caption}{}{\usepackage{caption}}
\AtBeginDocument{%
\ifdefined\contentsname
  \renewcommand*\contentsname{Table of contents}
\else
  \newcommand\contentsname{Table of contents}
\fi
\ifdefined\listfigurename
  \renewcommand*\listfigurename{List of Figures}
\else
  \newcommand\listfigurename{List of Figures}
\fi
\ifdefined\listtablename
  \renewcommand*\listtablename{List of Tables}
\else
  \newcommand\listtablename{List of Tables}
\fi
\ifdefined\figurename
  \renewcommand*\figurename{Figure}
\else
  \newcommand\figurename{Figure}
\fi
\ifdefined\tablename
  \renewcommand*\tablename{Table}
\else
  \newcommand\tablename{Table}
\fi
}
\@ifpackageloaded{float}{}{\usepackage{float}}
\floatstyle{ruled}
\@ifundefined{c@chapter}{\newfloat{codelisting}{h}{lop}}{\newfloat{codelisting}{h}{lop}[chapter]}
\floatname{codelisting}{Listing}
\newcommand*\listoflistings{\listof{codelisting}{List of Listings}}
\makeatother
\makeatletter
\makeatother
\makeatletter
\@ifpackageloaded{caption}{}{\usepackage{caption}}
\@ifpackageloaded{subcaption}{}{\usepackage{subcaption}}
\makeatother

\usepackage{bookmark}

\IfFileExists{xurl.sty}{\usepackage{xurl}}{} % add URL line breaks if available
\urlstyle{same} % disable monospaced font for URLs
\hypersetup{
  pdftitle={Functional Phytoplankton Group Retrieval From Space},
  pdfauthor={Susanne E. Craig; Erdem M. Karaköylü},
  pdfkeywords={Phytoplankton Functional Groups (PFGs), Hyperspectral
Ocean Color, Remote Sensing of Phytoplankton, Chlorophyll-a
Retrieval, PACE Mission},
  colorlinks=true,
  linkcolor={blue},
  filecolor={Maroon},
  citecolor={Blue},
  urlcolor={Blue},
  pdfcreator={LaTeX via pandoc}}


\journalname{Earth and Space Science}

\draftfalse

\begin{document}
\title{Functional Phytoplankton Group Retrieval From Space}

\authors{Susanne E. Craig\affil{1}, Erdem M. Karaköylü\affil{2}}
\affiliation{1}{NASA, }\affiliation{2}{Consultant, }
\correspondingauthor{Susanne E.
Craig}{steve@curvenote.com}\correspondingauthor{Erdem M.
Karaköylü}{erdemk@protonmail.com}


\begin{abstract}
In September 2021, a significant jump in seismic activity on the island
of La Palma (Canary Islands, Spain) signaled the start of a volcanic
crisis that still continues at the time of writing. Earthquake data is
continually collected and published by the Instituto Geográphico
Nacional (IGN). \ldots{}
\end{abstract}

\section*{Plain Language Summary}
Earthquake data for the island of La Palma from the September 2021
eruption is found \ldots{}




\section{Introduction}\label{introduction}

Phytoplankton functional groups (PFGs) play a fundamental role in marine
biogeochemical cycles, influencing carbon sequestration, nutrient
fluxes, and global climate feedbacks. Different functional groups
contribute uniquely to these processes; for example, diatoms facilitate
carbon export through rapid sinking, cyanobacteria fix atmospheric
nitrogen, and coccolithophores regulate carbonate chemistry via
calcification (Boyd \& Doney, 2010; Marañón et al., 2015). Identifying
and quantifying these groups from space is crucial for understanding
their ecological functions, detecting environmental changes, and
improving ocean biogeochemical models (Bopp et al., 2005; Laufkötter et
al., 2015). However, current satellite ocean-color products primarily
provide total chlorophyll-\emph{a} concentrations, which do not directly
indicate community composition. To address this gap, various remote
sensing algorithms have been developed to infer phytoplankton diversity,
each with limitations in distinguishing certain groups and quantifying
their biomass accurately (Mouw et al., 2017).

\subsection{Remote Sensing Approaches for PFG
Retrieval}\label{remote-sensing-approaches-for-pfg-retrieval}

Phytoplankton classification from satellite remote sensing has
traditionally relied on empirical and semi-analytical methods. Empirical
band-ratio techniques, such as PHYSAT, classify dominant phytoplankton
groups based on anomalies in spectral reflectance but are often
region-specific and limited to broad functional classes (Alvain et al.,
2005, 2008). Semi-analytical models, in contrast, use inherent optical
properties (IOPs) to infer phytoplankton composition from satellite
reflectance, providing a more mechanistic approach (Hirata et al.,
2011). Hybrid models incorporate additional environmental variables,
such as sea surface temperature and total chlorophyll, to infer
community structure (Brewin et al., 2010).

More recently, hyperspectral ocean-color sensors, such as NASA's
upcoming Plankton, Aerosol, Cloud, ocean Ecosystem (PACE) mission, have
been designed to improve PFG retrieval by capturing finer spectral
features associated with phytoplankton pigments (Dierssen et al., 2023).
These advancements offer the potential for enhanced classification, yet
challenges remain. Optical similarity between different groups,
depth-related biases in surface measurements, and uncertainties in
algorithm parameterization continue to limit retrieval accuracy (IOCCG,
2014). Most current models either estimate phytoplankton size classes or
assign a single dominant group per pixel, often failing to capture the
complexity of mixed communities (Ciotti et al., 2002).

\subsection{Study Contribution and
Approach}\label{study-contribution-and-approach}

In this study, we introduce \textbf{extreme gradient boosting (XGBoost)}
as a novel approach for retrieving phytoplankton functional groups from
satellite ocean color data. XGBoost is a scalable ensemble learning
algorithm that has demonstrated high performance in complex
classification tasks but has not yet been widely applied to PFG
retrieval (Chen \& Guestrin, 2016). Unlike traditional empirical and
semi-analytical models, our method leverages a large dataset of
simulated hyperspectral reflectance and environmental variables to
improve both the discrimination of functional groups and the
quantification of their biomass. Previous remote sensing algorithms
often classified only a dominant PFG or broad size class (Mouw et al.,
2017) and relied on empirical band relationships that lacked
generalizability (Hirata et al., 2011). By utilizing a machine learning
framework capable of integrating multiple features, our approach reduces
classification errors and enhances retrieval precision.

Recent studies have demonstrated the potential of XGBoost in related
applications, such as harmful algal bloom detection (Izadi et al., 2021)
and phytoplankton biomass estimation (Yan et al., 2025), highlighting
its suitability for remote sensing applications. Our work aligns with
the objectives of the PACE mission by contributing an advanced
classification algorithm that enhances hyperspectral monitoring of
phytoplankton diversity (Zhang et al., 2024). To our knowledge, this is
the first application of XGBoost for PFG classification in ocean color
remote sensing, offering a robust alternative to traditional retrieval
methods.

\section{Methods}\label{methods}

\subsection{Data Preparation and Feature
Selection}\label{data-preparation-and-feature-selection}

We utilized a simulated dataset representing the world ocean over 31
days, corresponding to December 2021. The simulation generated
hyperspectral remote sensing reflectance (Rrs) data, emulating a sensor
configuration akin to that of the PACE instrument. Due to the high
dimensionality of the original spectral data, we conducted an initial
exploratory analysis and observed strong correlations among many of the
channels. To reduce redundancy while preserving essential spectral
information, we retained 51 channels by selecting one channel every ten.
We opted against applying principal component analysis (PCA) because
preliminary investigations indicated that PCA tended to overemphasize
the signal from extensive blue water areas, which could mask coastal
processes of interest. In addition to these 51 spectral channels, we
included auxiliary environmental variables such as temperature and
latitude.

The dataset was divided into training and test sets using an 80/20
split. The training set was exclusively used for model development and
hyperparameter optimization, while the test set was reserved for the
final validation of model performance.

\subsection{Modeling and Hyperparameter
Optimization}\label{modeling-and-hyperparameter-optimization}

We employed an XGBoost Regressor model wrapped within a multi-output
regression structure to simultaneously predict multiple phytoplankton
functional groups as well as total chlorophyll-\emph{a} concentration.
XGBoost is a high-performance, scalable implementation of gradient
boosting that has become a popular choice for a wide range of regression
and classification tasks (Chen \& Guestrin, 2016). It builds an ensemble
of decision trees sequentially, where each new tree attempts to correct
the errors made by the previous trees. By optimizing a regularized
objective function, XGBoost effectively controls overfitting while
enhancing prediction accuracy. Its efficient handling of sparse data,
support for parallel computation, and flexible regularization mechanisms
make it particularly well-suited for complex modeling tasks, such as
predicting multiple phytoplankton functional groups and total
chlorophyll-\emph{a} concentration from hyperspectral and ancillary
oceanographic data.

Given the complexity of the problem and the high dimensionality of the
input features, it was critical to optimize the hyperparameters to
achieve robust performance and prevent overfitting. To this end, we
conducted hyperparameter optimization using a Bayesian optimization
approach(Snoek et al., 2012) implemented with Optuna(Akiba et al.,
2019). The objective function minimized the root mean squared error
(RMSE) computed via three-fold cross-validation on the training set. The
hyperparameters under investigation included the learning rate, maximum
tree depth, number of estimators, subsample ratio, column subsample
ratio, and gamma (the minimum loss reduction required to make a further
partition on a leaf node). The Bayesian optimization procedure allowed
us to efficiently explore the hyperparameter space by leveraging past
trial information to prune unpromising candidates early, thereby
reducing overall computational cost.

\subsection{Model Evaluation and eXplainable AI
(XAI)}\label{model-evaluation-and-explainable-ai-xai}

Once the optimal hyperparameter combination was identified, we retrained
the final XGBoost model on the full training set using these optimized
settings. Finally, we evaluated the performance of the retrained model
on the held-out test set to assess its generalizability.

\subsubsection{XAI with SHAP}\label{xai-with-shap}

To enhance interpretability and gain insights into how different input
features influence model predictions, we employed \textbf{Shapley
Additive Explanations (SHAP)}, a widely used explainable AI (XAI)
framework for interpreting complex machine learning models. SHAP is
named after the concept of Shapley values, which consists in assigning
importance values to each input feature by estimating its contribution
to the model's predictions across different samples. The method is
rooted in cooperative game theory, ensuring a fair distribution of
importance scores among features (Lundberg \& Lee, 2017).

Given the computational complexity of our XGBoost model and the high
dimensionality of the dataset, we conducted SHAP analysis on a
\textbf{random subsample of 1,000 observations from the test set}. This
subset was selected to balance computational feasibility while
maintaining a representative sample of phytoplankton spectral diversity.

We generated \textbf{SHAP summary plots}, which provide a comprehensive
visualization of feature importance and the directionality of their
influence on model outputs. These plots display the magnitude of each
feature's impact across all predictions, helping to identify the most
influential spectral and environmental variables in determining
phytoplankton functional group composition. The insights gained from
SHAP analysis aid in validating model behavior and ensuring its
ecological plausibility.

\paragraph{Code Availability}\label{code-availability}

All code used in this study is publicly available on GitHub ({[}GitHub
repository URL{]}).

\section{Results}\label{results}

\subsection{Hyperparameter Optimization
(HPO)}\label{hyperparameter-optimization-hpo}

We performed hyperparameter optimization using a Bayesian optimization
framework implemented with Optuna. The metric used for optimization was
the average RMSE (in units of \(mgL^{-1} Chl_a\) ) computed over the
cross-validation folds and across all target compartments. The ``full
HPO run'' best parameters indicate a relatively aggressive model,
characterized by deep trees with many estimators, a moderate learning
rate, and little regularization via gamma.

The best trial finished with an RMSE of \(0.116mgL^{-1} Chl_a\). Below
is the list of hyperparameters researched, the optimal values found, and
an interpretation of these values:

\begin{itemize}
\item
  Learning Rate (learning\_rate): \(0.083\) - This moderate learning
  rate suggests the model takes reasonably sized steps when updating
  that are neither too aggressive (which might lead to overshooting the
  optimum) nor too conservative (which could slow down convergence).
\item
  Max Depth (max\_depth): \(10\) - A depth of 10 allows the trees to
  capture complex interactions. This may indicate that the data has
  non-linear relationships that benefit from deeper trees. Such a depth
  can be associated with overfitting. The cross-validation process
  during HPO should minimize this however.
\item
  Number of Estimators (n\_estimators): \(466\) -Building around 466
  trees indicates the ensemble haa to tackle inherent complexity in the
  data that was not apparetn during the Exploratory Data Analysis phase.
  A larger number of trees generally improves performance---up to a
  point before overfitting becomse a risk. This number in conjunction
  with the cross validation process suggest this number strikes a
  balance between performance and overfitting.
\item
  Subsampling (subsample): \(0.658\) - This indicates each of the 466
  trees is using roughly 66\% of the data. This introduces randomness
  that helps prevent overfitting as not all samples in any
  cross-validation fold are used to build every tree.
\item
  Features used per tree (colsample\_bytree): \(0.894\) - Using about
  89\% of the features per tree indicates that most features are
  informative, and the model is allowed to consider almost the full
  feature set at each split. - See features used in the Methods section.
\item
  Gamma (gamma): \(8.63e-06\) - An extremely low gamma value means that
  almost no minimum loss reduction is required to make a split. This
  implies that the algorithm will split more readily, potentially
  capturing fine details. Awareness of this hyperparameter values is
  important as low gamma can risk overfitting.
\end{itemize}

\subsection{Optimized Model
Validation}\label{optimized-model-validation}

The next step was to load the best set of hyperparameter (listed above)
into the model and retrain the model on the entire training set. The
optimized and trained model was then validated using the test set, which
prior to the HPO process and until this step had been set aside . This
validation step ensured that the model performed satisfactorily on
unseen data and was ready for prediction on new data. The following
performance metrics were recorded during validation:

\begin{itemize}
\item
  \textbf{Mean Squared Error (MSE):}\\
  MSE is the average of the squared differences between the predicted
  and true values. Squaring the errors emphasizes larger deviations,
  making MSE sensitive to outliers. In our context, MSE is expressed in
  units of (mg\,L\(^{-1}\) Chl\(_a\))\(^2\). Lower MSE values indicate
  better model performance.
\item
  \textbf{Root Mean Squared Error (RMSE):}\\
  RMSE is the square root of the MSE, bringing the error metric back to
  the original units (mg\,L\(^{-1}\) Chl\(_a\)). It provides a direct
  measure of the average prediction error magnitude. Lower RMSE values
  suggest that the model's predictions are closer to the true values.
\item
  \textbf{Mean Absolute Error (MAE):}\\
  MAE calculates the average absolute difference between predicted and
  true values. Unlike MSE, it does not square the errors, so it is less
  sensitive to large outliers. MAE is also expressed in the same units
  as the target variable (mg\,L\(^{-1}\) Chl\(_a\)). A lower MAE
  indicates better predictive accuracy.
\item
  \textbf{Coefficient of Determination (R-squared):}\\
  R-squared measures the proportion of the variance in the dependent
  variable that is predictable from the independent variables. It ranges
  from 0 to 1, where a value closer to 1 indicates that the model
  explains a high proportion of the variance in the data. In our
  results, high R-squared values generally indicate strong model
  performance, although lower values (e.g., for dinoflagellates) suggest
  room for improvement.
\item
  \textbf{MAE/StDev\(_{true}\):}\\
  This ratio compares the mean absolute error to the standard deviation
  of the true values. It provides a relative measure of error by
  indicating how the average error compares to the inherent variability
  in the data. A lower ratio implies that the model's prediction error
  is small relative to the natural variability of the observations.
\end{itemize}

\subsection{XAI with Shapley Values}\label{xai-with-shapley-values}

The SHAP summary plots provides insights into feature importance and
their effects on model predictions for phytoplankton functional groups.
Features are ranked by impact, with the x-axis representing SHAP
values---positive values increase predicted chlorophyll concentration,
while negative values decrease it. The violin plot format shows the
distribution of SHAP values, with wider sections indicating greater
variability. The color gradient represents feature values, with red for
high values and blue for low values. The midpoint of the color bar
reflects a percentile-based central value, not necessarily the mean,
median, or mode, as it depends on the feature's distribution.

\subsection{Conclusion}\label{conclusion}

\subsection*{References}\label{references}
\addcontentsline{toc}{subsection}{References}

\phantomsection\label{refs}
\begin{CSLReferences}{1}{0}
\vspace{1em}

\bibitem[\citeproctext]{ref-Akiba2019}
Akiba, T., Sano, S., Yanase, T., Ohta, T., \& Koyama, M. (2019). Optuna:
A next-generation hyperparameter optimization framework. In
\emph{Proceedings of the 25th ACM SIGKDD international conference on
knowledge discovery and data mining (KDD)} (pp. 2623--2631). ACM.
\url{https://doi.org/10.1145/3292500.3330701}

\bibitem[\citeproctext]{ref-Alvain2005}
Alvain, S., Moulin, C., Dandonneau, Y., \& Bréon, F. M. (2005). Remote
sensing of phytoplankton groups in case 1 waters from global SeaWiFS
imagery. \emph{Deep-Sea Research I}, \emph{52}(11), 1989--2004.
\url{https://doi.org/10.1016/j.dsr.2005.06.015}

\bibitem[\citeproctext]{ref-Alvain2008}
Alvain, S., Moulin, C., Dandonneau, Y., \& Loisel, H. (2008). Seasonal
distribution and succession of dominant phytoplankton groups in the
global ocean: A satellite view. \emph{Global Biogeochemical Cycles},
\emph{22}(3), GB3S04. \url{https://doi.org/10.1029/2007GB003154}

\bibitem[\citeproctext]{ref-Bopp2005}
Bopp, L., Aumont, O., Cadule, P., Alvain, S., \& Gehlen, M. (2005).
Response of diatoms distribution to global warming and potential
implications: A global model study. \emph{Geophysical Research Letters},
\emph{32}(19), L19606. \url{https://doi.org/10.1029/2005GL023653}

\bibitem[\citeproctext]{ref-Boyd2010}
Boyd, P. W., \& Doney, S. C. (2010). Modelling regional responses by
marine pelagic ecosystems to global climate change. \emph{Geophysical
Research Letters}, \emph{32}(19), L19606.
\url{https://doi.org/10.1029/2005GL023653}

\bibitem[\citeproctext]{ref-Brewin2010}
Brewin, R. J. W., Sathyendranath, S., Hirata, T., Lavender, S. J.,
Barciela, R., \& Hardman-Mountford, N. J. (2010). A three-component
model of phytoplankton size class for the atlantic ocean.
\emph{Ecological Modelling}, \emph{221}(11), 1472--1483.
\url{https://doi.org/10.1016/j.ecolmodel.2010.02.014}

\bibitem[\citeproctext]{ref-Chen2016}
Chen, T., \& Guestrin, C. (2016). {XGBoost}: A scalable tree boosting
system. In \emph{Proceedings of the 22nd ACM SIGKDD international
conference on knowledge discovery and data mining (KDD)} (pp. 785--794).
\url{https://doi.org/10.1145/2939672.2939785}

\bibitem[\citeproctext]{ref-Ciotti2002}
Ciotti, A. M., Lewis, M. R., \& Cullen, J. J. (2002). Assessment of the
relationships between dominant cell size in natural phytoplankton
communities and the spectral shape of the absorption coefficient.
\emph{Limnology and Oceanography}, \emph{47}(2), 404--417.
\url{https://doi.org/10.4319/lo.2002.47.2.0404}

\bibitem[\citeproctext]{ref-Dierssen2023}
Dierssen, H. M., Gierach, M. M., Guild, L. S., Mannino, A., Salisbury,
J., Schollaert Uz, S., et al. (2023). Synergies between NASA's
hyperspectral aquatic missions PACE, GLIMR, and SBG: Opportunities for
new science and applications. \emph{Journal of Geophysical Research:
Biogeosciences}, \emph{128}(5), e2023JG007574.
\url{https://doi.org/10.1029/2023JG007574}

\bibitem[\citeproctext]{ref-Hirata2011}
Hirata, T., Hardman-Mountford, N. J., Brewin, R. J. W., Aiken, J.,
Barlow, R. G., Suzuki, K., et al. (2011). Synoptic relationships between
surface chlorophyll-a and diagnostic pigments specific to phytoplankton
functional types. \emph{Biogeosciences}, \emph{8}(2), 311--327.
\url{https://doi.org/10.5194/bg-8-311-2011}

\bibitem[\citeproctext]{ref-IOCCG2014}
IOCCG. (2014). \emph{Phytoplankton functional types from space} (IOCCG
reports No. No. 15). (S. Sathyendranath, Ed.). Dartmouth, Canada:
International Ocean-Colour Coordinating Group (IOCCG).

\bibitem[\citeproctext]{ref-Izadi2021}
Izadi, M., Sultan, M., El Kadiri, R., Ghannadi, A., \& Abdelmohsen, K.
(2021). A remote sensing and machine learning-based approach to forecast
the onset of harmful algal bloom. \emph{Remote Sensing}, \emph{13}(19),
3863. \url{https://doi.org/10.3390/rs13193863}

\bibitem[\citeproctext]{ref-Laufkotter2015}
Laufkötter, C., Vogt, M., Gruber, N., Aumont, O., Bopp, L., Doney, S.
C., et al. (2015). Projected decreases in future marine export
production: The role of the carbon flux through the upper ocean
ecosystem. \emph{Biogeosciences}, \emph{13}(13), 4023--4047.
\url{https://doi.org/10.5194/bg-13-4023-2016}

\bibitem[\citeproctext]{ref-Lundberg2017}
Lundberg, S. M., \& Lee, S.-I. (2017). A unified approach to
interpreting model predictions. In \emph{Proceedings of the 31st
international conference on neural information processing systems
(NeurIPS)} (Vol. 30, pp. 4768--4777). Curran Associates, Inc.
\url{https://doi.org/10.48550/arXiv.1705.07874}

\bibitem[\citeproctext]{ref-Maranon2015}
Marañón, E., Cermeno, P., Latasa, M., \& Tadonléké, R. D. (2015).
Resource supply overrides temperature as a controlling factor of marine
phytoplankton growth. \emph{PLoS ONE}, \emph{10}(6), e0130093.
\url{https://doi.org/10.1371/journal.pone.0130093}

\bibitem[\citeproctext]{ref-Mouw2017}
Mouw, C. B., Hardman-Mountford, N. J., Alvain, S., Bracher, A., Brewin,
R. J. W., Bricaud, A., et al. (2017). A consumer's guide to satellite
remote sensing of multiple phytoplankton groups in the global ocean.
\emph{Frontiers in Marine Science}, \emph{4}, 41.
\url{https://doi.org/10.3389/fmars.2017.00041}

\bibitem[\citeproctext]{ref-Snoek2012}
Snoek, J., Larochelle, H., \& Adams, R. P. (2012). Practical bayesian
optimization of machine learning algorithms. In \emph{Advances in neural
information processing systems (NeurIPS)} (Vol. 25, pp. 2951--2959).
Retrieved from
\url{https://proceedings.neurips.cc/paper/2012/hash/05311655a15b75fab86956663e1819cd-Abstract.html}

\bibitem[\citeproctext]{ref-Yan2025}
Yan, Z., Fang, C., Song, K., Wang, X., Wen, Z., Shang, Y., et al.
(2025). Spatiotemporal variation in biomass abundance of different algal
species in lake hulun using machine learning and sentinel-3 images.
\emph{Scientific Reports}, \emph{15}, 2739.
\url{https://doi.org/10.1038/s41598-025-87338-4}

\bibitem[\citeproctext]{ref-Zhang2024}
Zhang, Y., Shen, F., Li, R., Li, M., Li, Z., Chen, S., \& Sun, X.
(2024). {AIGD-PFT}: The first AI-driven global daily gap-free 4~km
phytoplankton functional type data product from 1998 to 2023.
\emph{Earth System Science Data}, \emph{16}, 4793--4816.
\url{https://doi.org/10.5194/essd-16-4793-2024}

\end{CSLReferences}




\end{document}
